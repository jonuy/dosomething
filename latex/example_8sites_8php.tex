\hypertarget{example_8sites_8php}{
\section{html/sites/example.sites.php File Reference}
\label{example_8sites_8php}\index{html/sites/example.sites.php@{html/sites/example.sites.php}}
}


\subsection{Detailed Description}
Configuration file for Drupal's multi-\/site directory aliasing feature.

This file allows you to define a set of aliases that map hostnames, ports, and pathnames to configuration directories in the sites directory. These aliases are loaded prior to scanning for directories, and they are exempt from the normal discovery rules. See \hyperlink{default_8settings_8php}{default.settings.php} to \hyperlink{classview}{view} how Drupal discovers the configuration directory when no alias is found.

Aliases are useful on development servers, where the domain name may not be the same as the domain of the live server. Since Drupal stores file paths in the database (files, system table, etc.) this will ensure the paths are correct when the site is deployed to a live server.

To use this file, copy and rename it such that its path plus filename is 'sites/sites.php'. If you don't need to use multi-\/site directory aliasing, then you can safely ignore this file, and Drupal will ignore it too.

Aliases are defined in an associative array named \$sites. The array is written in the format: '$<$port$>$.$<$domain$>$.$<$path$>$' =$>$ 'directory'. As an example, to map \href{http://www.drupal.org:8080/mysite/test}{\tt http://www.drupal.org:8080/mysite/test} to the configuration directory sites/example.com, the array should be defined as: 
\begin{DoxyCode}
 $sites = array(
   '8080.www.drupal.org.mysite.test' => 'example.com',
 );
\end{DoxyCode}
 The URL, \href{http://www.drupal.org:8080/mysite/test/,}{\tt http://www.drupal.org:8080/mysite/test/,} could be a symbolic link or an Apache Alias directive that points to the Drupal root containing \hyperlink{index_8php}{index.php}. An alias could also be created for a subdomain. See the \hyperlink{}{online Drupal installation guide } for more information on setting up domains, subdomains, and subdirectories.

The following examples look for a site configuration in sites/example.com: 
\begin{DoxyCode}
 URL: http://dev.drupal.org
 $sites['dev.drupal.org'] = 'example.com';

 URL: http://localhost/example
 $sites['localhost.example'] = 'example.com';

 URL: http://localhost:8080/example
 $sites['8080.localhost.example'] = 'example.com';

 URL: http://www.drupal.org:8080/mysite/test/
 $sites['8080.www.drupal.org.mysite.test'] = 'example.com';
\end{DoxyCode}


\begin{DoxySeeAlso}{See also}
\hyperlink{default_8settings_8php}{default.settings.php} 

\hyperlink{bootstrap_8inc_acef612ef19c49f6259531f0bee5c26cc}{conf\_\-path()} 

\href{http://drupal.org/documentation/install/multi-site}{\tt http://drupal.org/documentation/install/multi-\/site} 
\end{DoxySeeAlso}
