\hypertarget{group__views__handlers}{
\section{About Views handlers}
\label{group__views__handlers}\index{About Views handlers@{About Views handlers}}
}
In Views, a handler is an object that is part of the \hyperlink{classview}{view} and is part of the query building flow.

Handlers are objects; much of the time, the base handlers will work, but often you'll need to override the handler to achieve something meaningful. One typical handler override will be views\_\-handler\_\-filter\_\-operator\_\-in which allows you to have a filter select from a list of options; you'll need to override this to provide your list.

Handlers have two distinct code flows; the UI flow and the \hyperlink{classview}{view} building flow.

For the query flow:
\begin{DoxyItemize}
\item handler-\/$>$construct()
\begin{DoxyItemize}
\item Create the initial handler; at this time it is not yet attached to a \hyperlink{classview}{view}. It is here that you can set basic defaults if needed, but there will be no knowledge of the environment yet.
\end{DoxyItemize}
\item handler-\/$>$set\_\-definition()
\begin{DoxyItemize}
\item Set the data from \hyperlink{group__views__hooks_ga227057901681e4a33e33c199c7a8c989}{hook\_\-views\_\-data()} relevant to the handler.
\end{DoxyItemize}
\item handler-\/$>$init()
\begin{DoxyItemize}
\item Attach the handler to a \hyperlink{classview}{view}, and usually provides the options from the display.
\end{DoxyItemize}
\item handler-\/$>$pre\_\-query()
\begin{DoxyItemize}
\item Run prior to the query() stage to do early processing.
\end{DoxyItemize}
\item handler-\/$>$query()
\begin{DoxyItemize}
\item Do the bulk of the work this handler needs to do to add itself to the query.
\end{DoxyItemize}
\end{DoxyItemize}

Fields, being the only handlers concerned with output, also have an extended piece of the flow:


\begin{DoxyItemize}
\item handler-\/$>$pre\_\-render(\&\$values)
\begin{DoxyItemize}
\item Called prior to the actual rendering, this allows handlers to query for extra data; the entire resultset is available here, and this is where items that have \char`\"{}multiple values\char`\"{} per record can do their extra query for all of the records available. There are several examples of this at work in the code, see for example \hyperlink{classviews__handler__field__user__roles}{views\_\-handler\_\-field\_\-user\_\-roles}.
\end{DoxyItemize}
\item handler-\/$>$render()
\begin{DoxyItemize}
\item This does the actual work of rendering the field.
\end{DoxyItemize}
\end{DoxyItemize}

Most handlers are just extensions of existing classes with a few tweaks that are specific to the field in question. For example, \hyperlink{classviews__handler__filter__in__operator}{views\_\-handler\_\-filter\_\-in\_\-operator} provides a simple mechanism to set a multiple-\/value list for setting filter values. Below, \hyperlink{classviews__handler__filter__node__type}{views\_\-handler\_\-filter\_\-node\_\-type} overrides the list options, but inherits everything else.


\begin{DoxyCode}
 class views_handler_filter_node_type extends views_handler_filter_in_operator {
   function get_value_options() {
     if (!isset($this->value_options)) {
       $this->value_title = t('Node type');
       $types = node_get_types();
       foreach ($types as $type => $info) {
         $options[$type] = $info-&gt;name;
       }
       $this->value_options = $options;
     }
   }
 }
\end{DoxyCode}


Handlers are stored in their own files and loaded on demand. Like all other module files, they must first be registered through the module's info file. For example:


\begin{DoxyCode}
 name = Example module
 description = "Gives an example of a module."
 core = 7.x
 files[] = example.module
 files[] = example.install

 ; Views handlers
 files[] = includes/views/handlers/example_handler_argument_string.inc
\end{DoxyCode}


The best place to learn more about handlers and how they work is to explore \hyperlink{group__views__handlers}{Views' handlers } and use existing handlers as a guide and a model. Understanding how \hyperlink{classviews__handler}{views\_\-handler} and its child classes work is handy but you can do a lot just following these models. You can also explore the views module directory, particularly \hyperlink{node_8views_8inc}{node.views.inc}.

Please note that while all handler names in views are prefixed with views\_\-, you should use your own module's name to prefix your handler names in order to ensure namespace safety. Note that the basic pattern for handler naming goes like this:

\mbox{[}module\mbox{]}\_\-handler\_\-\mbox{[}type\mbox{]}\_\-\mbox{[}tablename\mbox{]}\_\-\mbox{[}fieldname\mbox{]}.

Sometimes table and fieldname are not appropriate, but something that resembles what the table/field would be can be used.

See also:
\begin{DoxyItemize}
\item \hyperlink{group__views__field__handlers}{Views field handlers }
\item \hyperlink{group__views__sort__handlers}{Views sort handlers }
\item \hyperlink{group__views__filter__handlers}{Views filter handlers }
\item \hyperlink{group__views__argument__handlers}{Views argument handlers }
\item \hyperlink{group__views__relationship__handlers}{Views relationship handlers }
\item \hyperlink{group__views__area__handlers}{Views area handlers } 
\end{DoxyItemize}